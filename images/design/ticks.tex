\documentclass[crop,tikz,convert={density=600,size=700x400,outext=.png}]{standalone}
\usetikzlibrary{patterns}

\begin{document}
%!TEX root = ../main.tex
\begin{tikzpicture}[scale=1,every node/.style={font=\scriptsize}]
\tikzset{>=latex}

\def\numTicks{2}
\def\tickWidth{4}

% \draw (0,0) to (1.5+\numTicks*\tickWidth,0);
% \node at (-.25,0) [left] {Time (ms)};
% \foreach \i in {0,...,5} {
%   \draw (\i*\tickWidth,.25) to (\i*\tickWidth,-.25);
  
%   \node at (\i*\tickWidth,-.25) [below] {\pgfmathparse{\i*250}\pgfmathprintnumber[1000 sep={}]{\pgfmathresult}};
% }

\node at (1,5) [left] {Ticks};
\node at (1,4) [left] {\texttt{TickListener} 1};
\node at (1,3) [left] {\texttt{TickListener} 2};
\node at (1,2) [left] {\texttt{TickListener} n};

\foreach \i in {1,...,\numTicks} {
  % \draw[fill=white] (\i*\tickWidth-\tickWidth,4.75) rectangle (\i*\tickWidth-\tickWidth+.5,5.25);  
  \node at (\tickWidth*.5+\i*\tickWidth-\tickWidth,5) [square,draw,fill=white] (n\i) {$t_{\i}$};

  \foreach \j in {2,3,4}{
    \node at (\tickWidth*.5+\i*\tickWidth-\tickWidth, \j) [square,draw,fill=white] (\i\j) {\texttt{tick(..)}};
  }

  \foreach \j in {3,4}{
    \pgfmathtruncatemacro{\tmp}{\j-1}
    \draw[->] (\i\j) to (\i\tmp);
  }

  \foreach \j in {2,3,4}{
    \node at (\tickWidth+\i*\tickWidth-\tickWidth, \j) [square,draw,fill=white] (a\i\j) {\texttt{afterTick(..)}};
  }
  \foreach \j in {3,4}{
    \pgfmathtruncatemacro{\tmp}{\j-1}
    \draw[->] (a\i\j) to (a\i\tmp);
  }
  
}

\draw[->] (n1) to (14);
\draw[->] (12) to (2, 1.5) to (2.85, 1.5) to (2.85, 4.5) to (4, 4.5) to (a14);
\draw[->] (a12) to (4, 1.5) to (5.15, 1.5) to (5.15, 5.5) to (6, 5.5) to (n2);
\draw[->] (n2) to (24);
\draw[->] (22) to (6, 1.5) to (6.85, 1.5) to (6.85, 4.5) to (8, 4.5) to (a24);
\draw[->,dashed] (a22) to (8, 1.5) to (9, 1.5);

\useasboundingbox ([shift={(0,-3\pgflinewidth)}]current 
bounding box.south west) -- (current bounding box.north east);

% (a14);



% \draw[fill=white] (6,.75) rectangle (7.5,1.25);
% \foreach \i in {5,...,7} {
  %\def\vara{\pgfmathparse{\i-5}\pgfmathprintnumber[fixed]{\pgfmathresult}}
  %\node at (0,0) {\pgfmathparse{\i-5}\pgfmathprintnumber[fixed]{\pgfmathresult}}
  %\def\vara{\i-5}
  %\draw (3.5+\i*0.5,.75) rectangle (3.5+\i*.5+.5,1.25);
%   \node at (3.5+\i*.5-0.07,1) [right] {$t_{\i}$};
% }
% \draw (6.5,.75) to (6.5,1.25);
% \draw (7,.75) to (7,1.25);


% \node at (-.25,2) [left] {\texttt{RtSolverModel} thread 0};

% \draw[fill=white] (.4,1.75) rectangle (2.5,2.25); 
% \draw[pattern=north west lines] (.4,1.75) rectangle (2.5,2.25); 
% \draw[->] (.4,1.25) to (.4,1.75);

% \draw[fill=white] (3.3,1.75) rectangle (4.1,2.25);  
% \draw[pattern=north west lines] (3.3,1.75) rectangle (4.1,2.25);  
% \draw[->] (3.3,1.25) to (3.3,1.75);

% \node at (-.25,3) [left] {\texttt{RtSolverModel} thread 1};

% \draw[fill=white] (.3,2.75) rectangle (4.4,3.25); 
% \draw[pattern=north west lines] (.3,2.75) rectangle (4.4,3.25); 
% \draw[->] (.3,1.25) to (.3,2.75);

%\draw[pattern=north west lines] (7.6,2.75) rectangle (9.5,3.25); 



\end{tikzpicture}
\end{document}